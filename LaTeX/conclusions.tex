\section{Concluding Remarks}

The study was able to present the design, development, and actual implementation of a two-staged automated green coffee bean sorting system, utilizing computer vision and embedded systems. The design is composed of a rotating conveyor table, a dual-camera inspection tray, defect sorting mechanism, and density-based sorting mechanism. In addition, five deep learning-based classification models such as EfficientNetV2, YOLOv8, YOLOv11, YOLOv12, and ViT were benchmarked. These models were deployed and tested into the actual defect sorting system with a test dataset of 20 beans per classification, where the ViT achieved the highest accuracy of 98\%. In terms of the sorting speed, the system was tested in 5 trials, where it achieved an average sorting speed of 22.2 beans per minute. The system was tested under varying quality distributions and maintained consistent sorting speeds, thereby confirming its practical viability. Overall, the results indicate that the integration of deep learning and embedded automation offers a robust and scalable solution for post-harvest coffee bean quality assessment.

\section{Contributions}
This study contributed to the coffee industry in the Philippines by introducing a two-stage automated coffee bean sorter that enhances coffee quality assessment by segregating defective beans and sorting dense and less-dense beans. This system integrates machine vision and density-based sorting, ensuring that high-quality, dense beans and potential specialty-grade coffee are selected for further processing. This system can support the Philippine coffee industry’s efforts to enhance product quality and meet global specialty coffee standards to improve market competitiveness. 


\section{Recommendations}
The following are the recommendations for further study of this design:
\begin{itemize}
    \item Optimize the density-based sorting mechanism
    \item Improvement of system portability by reducing the overall size and weight of the system
\end{itemize}

\section{Future Prospects}
This study offers a building block for future innovation in intelligent post-harvest coffee processing. A potential extension is combining cloud-based data storage and analytics for traceability at the batch level and remote monitoring. Another would be the deployment of light inference models on microcontroller units (MCUs) to facilitate real-time, on-device computation, thus minimizing system latency and increasing portability. Additional research might also investigate the use of unsupervised or semi-supervised learning methods to identify new or infrequent defects without depending solely on labeled data. Commercially, the system can be scaled to process greater volumes using modular conveyor lines and parallel sorting stations. These developments would greatly benefit coffee producers by providing consistent, efficient, and objective bean quality assessment.
