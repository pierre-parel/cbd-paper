\chapter*{Abstract}

With the growing demand for coffee, the Philippine coffee industry faces the challenge of inefficient and utilizing traditional manual sorting methods which hinders quality consistency and competitiveness. This study proposes to develop a two-stage automated green coffee bean (GCB) sorter designed to identify and segregate defective beans from the batch, and the good classification would proceed to the second stage which is the density-analysis stage wherein it sorts the dense and less-dense beans. The system integrates machine vision that utilizes the RF-DETR model for detection and uses YOLOv12 and Vision Transformer (ViT) models for classification. YOLOv12 achieved an average accuracy of 94.8\% across all trials with varying test conditions, while the ViT model achieved an accuracy of 98\%, which was similar to its per-classification accuracy. Overall, the ViT model was superior across all metrics in both training and testing phases. The density-analysis stage achieved an average sorting accuracy of 90.89\%, demonstrating its effectiveness in distinguishing dense beans from less-dense ones with high reliability. Overall, this automated sorter provides a reliable alternative to manual processing, which significantly reduces human labor and improves GCB quality, thereby strengthening the local coffee farmers and industry’s foundation and competitiveness.

